\documentclass{article}
  
\usepackage{amsthm}
\theoremstyle{plain} 
   \newtheorem{theorem}{Theorem}[section]
   \newtheorem{corollary}[theorem]{Corollary}
   \newtheorem{lemma}[theorem]{Lemma}
   \newtheorem{proposition}[theorem]{Proposition}
\theoremstyle{definition}
   \newtheorem{definition}[theorem]{Definition}
   \newtheorem{example}[theorem]{Example}
\theoremstyle{remark}    
  \newtheorem{remark}[theorem]{Remark}

\usepackage{amsfonts}
\usepackage{amsmath}
\usepackage{amssymb}
\usepackage{fullpage}
\usepackage{parskip}
\usepackage{graphicx}

\usepackage{hyperref}
  \hypersetup{
    colorlinks = true,
    urlcolor = blue,
    linkcolor= blue,
    citecolor= blue,
    filecolor= blue,
    }

\title{The MU Puzzle}
\author{Tommy Golbranson\\ Chapman University}
\date{\today}

\begin{document}

\maketitle

\begin{abstract}
This document will attempt to prove the impossibility of the MU-Puzzle from Chapter 1 of \href{https://www.physixfan.com/wp-content/files/GEBen.pdf}{AN ETERNAL GOLDEN BRAID}.
\end{abstract}

\tableofcontents

\section{Introduction}

The MU puzzle is a challenge to form the string $MU$ from the starting string $MI$, following a set of rules:

\begin{enumerate}
    \item \textbf{Rule I:} If you possess a string whose last letter is $I$, you can add on a $U$ at the end. 
    \item \textbf{Rule II:} Suppose you have $Mx$. Then you may add $Mxx$ to your collection. 
    \item \textbf{Rule III:} If $III$ occurs in one of the strings in your collection, you may make a new
    string with $U$ in place of $III$.
    \item \textbf{Rule IV:} If $UU$ occurs inside one of your strings, you can drop it.
\end{enumerate}
This puzzle is an interesting exercise of formulating proofs and theorems, and as the book explains, is more about the process and thinking through it than the solution itself.

\section{Analysis}
From my understanding of the puzzle, the only way to achieve $MU$ is by eliminating the starting $I$ or getting a string of $I$'s that is divisibile by 3, i.e. for $Mxx..x$, $x mod 3 = 0$.

\begin{lemma}
Rule II allows the string after $M$ to be doubled, but for any string of $n$ $I$'s, the prime factorization of $n$ is always 2*2*...*2.
\end{lemma}

\begin{proof}
Repeating Rule 2, we get:

\begin{enumerate}
  \item \textbf{Step 1:} $MI$
  \item \textbf{Step 2:} $MII$ (2*2)
  \item \textbf{Step 3:} $MIIII$ (2*2*2)
  \item \textbf{Step 4:} $MIIIIIIII$ (2*2*2*2)
  \item \textbf{etc...}
\end{enumerate}

Rule III allows a U to be exchanged for a string of $III$, but there will always be an $I$ leftover, as seen in the steps above.

Rules I and IV allow the number of $U$'s to vary, but this will never affect the number of $I$'s in the string.

Therefore, it is impossible to achieve $MU$ from $MI$ by the set of rules.

  
\end{proof}

\section{Main Result}
\begin{theorem}
As shown, it is impossible to achieve $MU$ from $MI$ by the set of rules allowed. The string of $I$'s will never be divisible by 3 without remainder, so there will always be at least 1 $I$ in the string. Therefore $MU$, which has no $I$, is impossible to achieve.
\end{theorem}

\begin{remark}
$MU$ is only achievable by a different starting condition, a different set of rules, or both.
\end{remark}

\section{Conclusion}
The goal of the puzzle was to achieve $MU$ from a starting string $MI$ which has been shown to be impossible. The key to this puzzle was the number of $I$'s never being divisible by 3. Since $MU$ is only achievable by eliminating all $I$'s, the goal is impossible.

\begin{thebibliography}{99}
\bibitem{Hofstadter} D. Hofstadter, \textit{Gödel, Escher, Bach: An Eternal Golden Braid}, Basic Books, 1979.
\end{thebibliography}

\end{document}
